\documentclass[a4paper,11pt]{article}

% --- Chargement des paquets de base ---
\usepackage[utf8]{inputenc}
\usepackage[T1]{fontenc}
\usepackage[french]{babel}
\usepackage[top=2cm, bottom=2cm, left=2cm, right=2cm]{geometry} % Marges optimisées pour les 15 pages max
\usepackage{lmodern}
\usepackage{microtype} % Améliore la typographie
\usepackage{graphicx}
\usepackage{url}
\usepackage[hidelinks]{hyperref}
\usepackage{enumitem}
\usepackage{booktabs} % Pour de beaux tableaux
\usepackage{longtable} % Pour les tableaux sur plusieurs pages
\usepackage{array}
\usepackage{pdflscape} % Pour mettre le tableau d'enseignement en paysage
\usepackage{fancyhdr}
\usepackage{fontawesome5} % Pour les icônes (email, web, etc.)
\usepackage{tcolorbox} % Pour les boites de mise en valeur
\usepackage{colortbl}


\usepackage{xcolor}
% Copie de vos définitions pour compilation immédiate
\definecolor{cornflower}{rgb}{0.55,0.79,0.90} 
\definecolor{blueGreen}{rgb}{0.12,0.61,0.73} 
\definecolor{oxfordBlue}{rgb}{0,0.18,0.27} 
\definecolor{tiffanyBlue}{rgb}{0,0.77,0.69}
\definecolor{uclaGold}{rgb}{1,0.71,0} 
\definecolor{orange}{rgb}{1,0.52,0} 
\definecolor{sunny}{rgb}{0.97,0.96,0.44}
\definecolor{deepSpaceSparkle}{rgb}{0.30, 0.37, 0.42}
\definecolor{gainsboro}{rgb}{0.86,0.86,0.86} 
\definecolor{whiteLavender}{rgb}{0.90,0.90,0.98} 
\definecolor{heliotropeMagenta}{rgb}{0.69,0.09,0.78} 
\definecolor{shockingPink}{rgb}{1.0,0.41,0.92} 
\definecolor{jellyBean}{rgb}{0.85,0.31,0.29} 
\definecolor{bittersweet}{rgb}{1.0,0.43,0.34} 
\definecolor{temptress}{rgb}{0.24,0.12,0.19}
\definecolor{shadow}{rgb}{0.54,0.44,0.33}
\definecolor{teaGreen}{rgb}{0.85,0.94,0.7} 
\definecolor{apple}{rgb}{0.41,0.71,0.27} 
\definecolor{fernGreen}{rgb}{0.35,0.47,0.28}

% --- Personnalisation des titres ---
\usepackage{titlesec}

% Titre de section : OxfordBlue, souligné
\titleformat{\section}
{\color{oxfordBlue}\normalfont\Large\bfseries}
{\thesection}{1em}{}[\titlerule]

% Titre de sous-section : BlueGreen
\titleformat{\subsection}
{\color{blueGreen}\normalfont\large\bfseries}
{\thesubsection}{1em}{}

% Titre de sous-sous-section : DeepSpaceSparkle
\titleformat{\subsubsection}
{\color{deepSpaceSparkle}\normalfont\normalsize\bfseries}
{\thesubsubsection}{1em}{}

% --- Configuration des listes ---
\setlist[itemize]{label=\color{uclaGold}\textbullet, leftmargin=*}

% --- En-tête et pied de page ---
\pagestyle{fancy}
\fancyhf{}
\lhead{\color{deepSpaceSparkle}\small Candidature Qualification MCF - Section 27}
\rhead{\color{deepSpaceSparkle}\small \textbf{Prénom NOM}}
\cfoot{\thepage}
\renewcommand{\headrulewidth}{0.4pt}
\renewcommand{\headrule}{\hbox to\headwidth{\color{cornflower}\leaders\hrule height \headrulewidth\hfill}}

% --- Commandes utiles ---
\newcommand{\cnuKey}[1]{\textbf{\textcolor{oxfordBlue}{#1}}} % Pour les mots clés CNU
\newcommand{\publiTag}[1]{\textcolor{jellyBean}{\textbf{[#1]}}} % Pour tagger les 3 publis majeures

\begin{document}

% --- TETE DU DOCUMENT ---
\begin{center}
    {\huge \textbf{\textcolor{oxfordBlue}{Dossier de candidature à la qualification MCF}}}\\
    \vspace{0.2cm}
    {\Large Section 27 - Informatique}\\
    \vspace{0.5cm}
\end{center}

\begin{tcolorbox}[colback=whiteLavender,colframe=blueGreen,title=Identité du candidat]
    \begin{tabular}{p{0.6\textwidth} p{0.35\textwidth}}
        \textbf{\Large Prénom NOM} &  \\
        Actuellement : \textit{Poste actuel (ex: ATER, Post-doctorant)} & \faEnvelope \ \href{mailto:votre.email@univ.fr}{votre.email@univ.fr} \\
        Laboratoire : \textit{Nom du Labo, Université} & \faGlobe \ \href{https://votre-site-web.fr}{votre-site-web.fr} \\
        & \faPhone \ +33 6 00 00 00 00
    \end{tabular}
\end{tcolorbox}

% --- DOMAINES DE RECHERCHE (IMPORTANT POUR LE CNU) ---
\section*{Domaines de recherche (Nomenclature CNU 27)}
\noindent
\textbf{Domaine principal :} \cnuKey{Code X - Intitulé Principal} \\
\textbf{Domaines secondaires :} Code Y - Intitulé ; Code Z - Intitulé. \\
\textbf{Mots-clés :} \textit{Intelligence Artificielle, Génie Logiciel, Réseaux, ...}

% --- 1. PARCOURS UNIVERSITAIRE ---
\section{Parcours Universitaire}

\begin{description}
    \item[20XX] \textbf{Doctorat en Informatique} -- \textit{Université de XXX}
    \begin{itemize}
        \item \textbf{Titre :} \textit{"Titre de la thèse"}
        \item \textbf{Directeur(s) :} Prénom Nom (Labo), Prénom Nom (Labo)
        \item \textbf{Rapporteurs :} Prénom Nom (Affiliation), Prénom Nom (Affiliation)
        \item \textbf{Jury :} Prénom Nom (Président), Prénom Nom (Examinateur)...
        \item \textit{Mention (si applicable) / Prix de thèse éventuel.}
    \end{itemize}
    
    \item[20XX] \textbf{Master en Informatique} (ou Diplôme d'ingénieur) -- \textit{Université de XXX}
    \begin{itemize}
        \item Spécialité XXX.
        \item \textbf{Stage de M2 :} "Titre", encadré par XXX.
    \end{itemize}
    
    \item[20XX] \textbf{Licence Informatique} -- \textit{Université de XXX}
\end{description}

% --- 2. EXPERIENCES PROFESSIONNELLES ---
\section{Expériences Professionnelles}

\begin{description}
    \item[20XX -- 20XX] \textbf{Intitulé du poste} (ATER, Post-doc, Ingénieur...) -- \textit{Lieu/Établissement}
    \begin{itemize}
        \item Statut : CDD / CDI.
        \item Description succincte : \textit{Recherche sur le projet X, Enseignement en L1/L2...}
    \end{itemize}
    
    \item[20XX -- 20XX] \textbf{Doctorant Contractuel} -- \textit{Lieu/Établissement}
    \begin{itemize}
        \item Mission d'enseignement (64h/an) ou Recherche seule.
    \end{itemize}
\end{description}

% --- 3. ACTIVITÉS D'ENSEIGNEMENT ---
\section{Activités d'enseignement}

\subsection{Synthèse des enseignements réalisés}

% Le tableau est large, on utilise small pour que ça rentre, ou landscape
% La consigne demande explicitement ces colonnes.
\begin{footnotesize}
\begin{longtable}{|p{1.2cm}|c|p{1.8cm}|p{1.2cm}|c|p{2cm}|c|c|c|p{2.5cm}|}
    \hline
    \rowcolor{oxfordBlue} 
    \textcolor{white}{\textbf{Statut}} & 
    \textcolor{white}{\textbf{Année}} & 
    \textcolor{white}{\textbf{Établissement}} & 
    \textcolor{white}{\textbf{Public}} & 
    \textcolor{white}{\textbf{Niv.}} & 
    \textcolor{white}{\textbf{Matière}} & 
    \textcolor{white}{\textbf{H (eqTD)}} & 
    \textcolor{white}{\textbf{Eff.}} & 
    \textcolor{white}{\textbf{Nature}} & 
    \textcolor{white}{\textbf{Responsabilités}} \\
    \hline
    \endhead
    
    % --- Ligne Exemple ---
    ATER & 2023-24 & Univ. Nantes & Info & L1 & Algo & 32 & 150 & TP & Création sujets \\
    \hline
    Doc & 2022-23 & Univ. Lyon & MIAGE & M1 & Base de données & 24 & 30 & TD & Resp. module \\
    \hline
    % --- Ajoutez vos lignes ici ---
    
    \hline
    \multicolumn{6}{|r|}{\textbf{TOTAL}} & \textbf{XXX h} & & & \\
    \hline
\end{longtable}
\end{footnotesize}

\subsection{Description des enseignements et démarche pédagogique}
\textit{(5 à 15 lignes recommandées)}
Décrivez ici votre approche. 
\begin{itemize}
    \item \textbf{Matière A :} Explication du contenu, innovations pédagogiques (classe inversée, projet, etc.), supports créés (lien vers GitHub ou Moodle si pertinent).
    \item \textbf{Matière B :} ...
    \item \textbf{Outils :} Utilisation de Docker, Jupyter Notebooks, plateformes de correction automatique...
\end{itemize}

\subsection{Projet d'enseignement}
\textit{(Texte synthétique précisant ce que vous êtes prêt à assurer)}
Compte tenu de mon parcours en [Domaine], je suis en mesure d'intervenir sur les enseignements de socle commun (Algorithmique, Programmation Objet, Bases de Données) ainsi que sur des enseignements spécialisés tels que [Vos Spécialités] en Master.
Je souhaite développer... [Votre vision].

% --- 4. ACTIVITÉS DE RECHERCHE ---
\section{Activités de recherche}

\subsection{Thématiques et Résultats majeurs}
\textit{Ne faites pas seulement une liste, racontez une histoire scientifique.}
Mes travaux de recherche se situent à l'intersection de A et B. Ils visent à résoudre le problème Z.

\begin{itemize}
    \item \textbf{Axe 1 : Titre de l'axe.} Description des résultats théoriques ou méthodologiques. \textit{Lien avec publications [1, 3]}.
    \item \textbf{Axe 2 : Titre de l'axe.} Description des expérimentations, mesures, évaluations. \textit{Lien avec publications [2]}.
\end{itemize}

\subsection{Production logicielle et Valorisation}
\begin{itemize}
    \item \textbf{Nom du Logiciel} : Description. URL du dépôt. (Licence, langage, impact).
    \item Participation à des projets open-source, brevets, jeux de données mis à disposition.
\end{itemize}

\subsection{Encadrement doctoral et scientifique}
\textit{Précisez votre taux d'encadrement et le devenir des étudiants.}
\begin{itemize}
    \item \textbf{2023 (6 mois) - Stage M2} : Prénom Nom, \textit{"Sujet"}. Co-encadrement à 50\% avec X. Résultat : Publication [4].
    \item \textbf{2022 (4 mois) - PFE Ingénieur} : Prénom Nom.
\end{itemize}

\subsection{Rayonnement et Visibilité}
\begin{itemize}
    \item \textbf{Organisation :} Membre du comité d'organisation de la conférence X.
    \item \textbf{Comités de programme (PC) :} Reviewer pour Conf X, Conf Y.
    \item \textbf{Collaborations :} Séjour scientifique au Labo Z (Pays), collaboration avec l'entreprise W.
    \item \textbf{Médiation :} Fête de la science, interventions en lycée.
\end{itemize}

% --- 5. PUBLICATIONS ---
\section{Liste des publications}

\begin{tcolorbox}[colback=sunny!20, colframe=uclaGold, title=Note aux rapporteurs]
Les \textbf{3 publications jointes} au dossier sont signalées par le symbole \publiTag{PDF}.
Les taux d'acceptation et les indicateurs de notoriété (Core, Quartile) sont précisés pour chaque entrée.
\end{tcolorbox}

% Utilisez cette structure pour chaque publi pour respecter les demandes du CNU 27
% Arguments : Auteurs, Année, Titre, Conférence/Revue, Type (Long/Court), Pages, Preuves/Métriques

\subsection*{Revues internationales avec comité de lecture (RI)}

\begin{enumerate}
    \item \textbf{V. Nom}, A. Autre. (2023). \textit{Titre de l'article de revue}. \textbf{Nom de la Revue (Journal)}, Vol X, pp 10-24.
    \\ \textcolor{deepSpaceSparkle}{\small \faInfoCircle \ Facteur d'impact : 3.5 (Q1). 20 pages. DOI: xxx.xxx}
    \\ \publiTag{PDF} \textit{Contribution : description de votre apport spécifique (obligatoire).}
\end{enumerate}

\subsection*{Conférences internationales avec comité de lecture (C-ACTI)}

\begin{enumerate}[resume]
    \item \textbf{V. Nom}, B. Collegue. (2022). \textit{Titre de l'article de conf}. Dans les actes de \textbf{Name of Conference (CONF 2022)}, Ville, Pays.
    \\ \textcolor{deepSpaceSparkle}{\small \faInfoCircle \ CORE A. Taux d'acceptation : 18\%. Article long (12 pages). Lien vers les actes : http...}
    \\ \publiTag{PDF} \textit{Contribution : J'ai réalisé l'implémentation et les preuves théoriques...}
    
    \item C. Auteur, \textbf{V. Nom}. (2021). ...
\end{enumerate}

\subsection*{Conférences nationales avec comité de lecture (C-ACTN)}
...

\subsection*{Autres (Workshops, Poster, Brevets, Thèse)}
...

% --- 6. RESPONSABILITÉS COLLECTIVES ---
\section{Responsabilités collectives et administratives}

\begin{itemize}
    \item \textbf{2021 -- 2023 :} Représentant des doctorants au conseil de laboratoire. (Estimation : 10h/an).
    \item \textbf{2022 :} Responsable de la mise en place du serveur de calcul de l'équipe.
\end{itemize}

% --- 7. ELEMENTS COMPLEMENTAIRES ---
\section{Éléments complémentaires}
\textit{Expliquez ici les trous dans le CV (congés parentaux, maladie), les parcours atypiques ou tout ce qui valorise le dossier.}

\vspace{1cm}
\begin{center}
    \textit{Document rédigé le \today.}
\end{center}

\end{document}